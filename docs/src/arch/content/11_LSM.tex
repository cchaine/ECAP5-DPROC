\section{Load-Store Module}

  \subsection{Interface}

    \begin{content}
        The load-store module implements TBC. The signals are described in table \ref{tab:lsm-interface}. 
      \end{content}

    {
  \vspace{0.5em}
  \begin{center}
    \refstepcounter{table}
    Table \thetable: Load-Store Module interface signals\label{tab:lsm-interface}
  \end{center}

\footnotesize
\begin{xltabular}{0.9\textwidth}{|l|c|c|X|}
  \hline
  \cellcolor{gray!20}\textbf{NAME} & \cellcolor{gray!20}\textbf{TYPE} & \cellcolor{gray!20}\textbf{WIDTH} & \cellcolor{gray!20}\textbf{DESCRIPTION} \\
  \hline
  clk\_i & I & 1 & Clock input. \\
  \hline
  rst\_i & I & 1 & Reset input. \\
  \hline
  \multicolumn{4}{|l|}{\textbf{INPUT LOGIC}} \\
  \hline
  input\_ready\_o & O & 1 & Input handshaking signal asserted when ready to receive inputs. \\
  \hline
  input\_valid\_i & I & 1 & Input handshaking signal asserted when the provided inputs are valid. \\
  \hline
  alu\_result\_i & I & 32 & ALU result \\
  \hline
  enable\_i & I & 1 & Bypass enable signal. The load-store unit is bypassed when asserted. \\
  \hline
  write\_i & I & 1 & Write enable. The memory is written when asserted and read when deasserted. \\
  \hline
  write\_data\_i & I & 32 & Data to be written in memory \\
  \hline
  sel\_i & I & 4 & Byte selection enable \\
  \hline
  reg\_write\_i & I & 1 & Register write enable \\
  \hline
  reg\_addr\_i & I & 5 & Register write address \\
  \hline
  \multicolumn{4}{|l|}{\textbf{WISHBONE MASTER}} \\
  \hline
  wb\_adr\_o & O & 32 & The address output array is used to pass binary address. \\
  \hline
  wb\_dat\_i & I & 32 & The data input array is used to pass binary data. \\
  \hline
  wb\_dat\_o & I & 32 & The data output array is used to pass binary data. \\
  \hline
  wb\_we\_o & O & 1 & The write enable output indicates whether the current local bus cycle is a READ or WRITE cycle. This signal is negated during READ cycles and is asserted during WRITE cycles. This signal is always low as this interface only supports READ cycles. \\
  \hline
  wb\_sel\_o & O & 4 & The select output array indicates where valid data is expected on the wb dat i signal array during READ cycles. Each individual select signal correlates to one of four active bytes on the 32-bit data port. \\
  \hline
  wb\_stb\_o & O & 1 & The strobe output indicates a valid data transfer cycle. It is used to qualify various other signals on the interface. \\
  \hline
  wb\_ack\_i & I & 1 & The acknowledge input, when asserted, indicates the nor- mal termination of a bus cycle. \\
  \hline
  wb\_cyc\_o & O & 1 & The cycle output, when asserted, indicates that a valid bus cycle is in progress. This signal is asserted for the duration of all bus cycles. \\
  \hline
  wb\_stall\_i & I & 1 & The pipeline stall input indicates that current slave is not able to accept the transfer in the transaction queue. \\
  \hline
  \multicolumn{4}{|l|}{\textbf{OUTPUT LOGIC}} \\
  \hline
  output\_valid\_o & O & 1 & Output handshaking signal asserted when the output is valid. \\
  \hline
  reg\_write\_o & O & 1 & Asserted when the instruction shall store its result. \\
  \hline
  reg\_addr\_o & O & 5 & Address of the register where the result shall be stored. \\
  \hline
  reg\_data\_o & O & 32 & Instruction result. \\
  \hline
\end{xltabular}
}


  \subsection{Specification}

    \subsubsection{State machine}

      \begin{figure}[H]
          \centering
          \vspace{1em}
\begin{tikzpicture}[->, >=stealth, node distance=3cm, every state/.style={thick, fill=gray!20}, initial text=$ $]
    \node[state,minimum width=2cm,initial]  (idle)            {\texttt{IDLE}};
    \node[state,minimum width=2cm, align=center]          (memory-stall)  [below right of=idle]  {\footnotesize{\texttt{MEMORY}} \\ \footnotesize{\texttt{STALL}}};
    \node[state,minimum width=2cm, align=center]          (request)       [above right of=memory-stall]  {\footnotesize{\texttt{REQUEST}}};
    \node[state,minimum width=2cm, align=center]          (memory-wait)   [below right of=request]  {\footnotesize{\texttt{MEMORY}} \\ \footnotesize{\texttt{WAIT}}};
    \node[state,minimum width=2cm, align=center]          (done)          [above right of=memory-wait]  {\texttt{DONE}};

  \draw (idle)            edge              (memory-stall)  
        (idle)            edge              (request)                         
        (memory-stall)    edge              (request)                         
        (request)         edge              (memory-wait)                     
        (request)         edge              (done)                            
        (memory-wait)     edge              (done)                            
        (done)            edge[bend right]  (idle);
\end{tikzpicture}

          \caption{State diagram of the load-store module}
          \label{fig:lsm-state-diagram}
        \end{figure}

      \req{D\_LSM\_STATE\_01}{
          The LSM shall be in the \texttt{IDLE} (\texttt{0}) state when there are no pending memory requests.
        }

      \req{D\_LSM\_STATE\_02}{
          The LSM shall be in the \texttt{REQUEST} (\texttt{1}) state when a memory request is successfully sent.
        }

      \req{D\_LSM\_STATE\_03}{
          The LSM shall be in the \texttt{MEMORY\_WAIT} (\texttt{2}) state when a memory request has been sent but the response has not been received.
        }

      \req{D\_LSM\_STATE\_04}{
          The LSM shall be in the \texttt{DONE} (\texttt{3}) state when a memory request response is successfully received.
        }

      \req{D\_LSM\_STATE\_05}{
          The LSM shall be in the \texttt{MEMORY\_STALL} (\texttt{4}) state when a memory request is ready but the memory interface is stalled.
        }

    \subsubsection{Input handshake}

      \req{D\_LSM\_INPUT\_READY\_01}{
          The \texttt{input\_ready\_o} signal shall be asserted while in the \texttt{IDLE} state.
        }

      \req{D\_LSM\_INPUT\_READY\_02}{
          The \texttt{input\_ready\_o} signal shall be deasserted on the rising edge of \texttt{clk\_i} after leaving the \texttt{IDLE} state.
        }

      \req{D\_LSM\_INPUT\_READY\_03}{
          The \texttt{input\_ready\_o} signal shall be deasserted on the rising edge of \texttt{clk\_i} when \texttt{rst\_i}.
        }

    \subsubsection{Request triggering}
      
      \event{EV\_LSM\_REQUEST\_TRIGGER\_01}{
          This event represents a memory request trigger request
        }

      \req{D\_LSM\_REQUEST\_TRIGGER\_01}{
          The \eventref{EV\_LSM\_REQUEST\_TRIGGER\_01} event shall be triggered on the rising edge of \texttt{clk\_i} while in the \texttt{IDLE} state when both \texttt{enable\_i} and \texttt{input\_valid\_i} are asserted.
        }

      \req{D\_LSM\_READ\_REQUEST\_01}{
          A memory read cycle with address \texttt{alu\_result\_i} shall be triggered when \eventref{EV\_LSM\_REQUEST\_TRIGGER\_01} is triggered and \texttt{write\_i} is deasserted.
        }

      \req{D\_LSM\_WRITE\_REQUEST\_01}{
        A memory write cycle with address \texttt{alu\_result\_i} and data \texttt{write\_data\_i} shall be triggered when \eventref{EV\_LSM\_REQUEST\_TRIGGER\_01} is triggered and \texttt{write\_i} is deasserted.
        }

      \req{D\_LSM\_REQUEST\_SIZE\_01}{
          The \texttt{wb\_sel\_o} signal shall be set to the value of \texttt{sel\_i} when \eventref{EV\_LSM\_REQUEST\_TRIGGER\_01} is triggered.
        }

    \subsubsection{Wishbone interface}

      \begin{content}
          The following requirements are extracted from the Wishbone specification for implementing the memory interface of the load-store module.
        \end{content}

      \req{D\_LSM\_WISHBONE\_DATASHEET\_01}{
          The memory interface shall comply with the Wishbone Datasheet provided in section \ref{user-needs}.
        }

      \req{D\_LSM\_WISHBONE\_RESET\_01}{
          The memory interface shall initialize itself at the rising edge of \texttt{clk\_i} following the assertion of \texttt{rst\_i}.
        }

      \req{D\_LSM\_WISHBONE\_RESET\_02}{
          The memory interface shall stay in the initialization state until the rising edge of \texttt{clk\_i} following the deassertion of \texttt{rst\_i}.
        }

      \req{D\_LSM\_WISHBONE\_RESET\_03}{
          Signals \texttt{wb\_stb\_o} and \texttt{wb\_cyc\_o} shall be deasserted while the memory interface is in the initialization state. The state of all other memory interface signals are undefined in response to a reset cycle.
        }

      \req{D\_LSM\_WISHBONE\_TRANSFER\_CYCLE\_01}{
          The memory interface shall assert \texttt{wb\_cyc\_o} for the entire duration of the memory access.
        }[
          rationale=TBC what wb\_cyc\_o does.
        ]

      \req{D\_LSM\_WISHBONE\_TRANSFER\_CYCLE\_02}{
          Signal \texttt{wb\_cyc\_o} shall be asserted no later than the rising edge of \texttt{clk\_i} that qualifies the assertion of \texttt{wb\_stb\_o}.
        }

      \req{D\_LSM\_WISHBONE\_TRANSFER\_CYCLE\_03}{
            Signal \texttt{wb\_cyc\_o} shall be deasserted no earlier than the rising edge of \texttt{clk\_i} that qualifies the deassertion of \texttt{wb\_stb\_o}.
        }

      \req{D\_LSM\_WISHBONE\_HANDSHAKE\_01}{
          The memory interface shall accept \texttt{wb\_ack\_i} signals at any time after a transaction is initiated.
        }

      \req{D\_LSM\_WISHBONE\_HANDSHAKE\_02}{
          The memory interface must qualify the following signals with \texttt{wb\_stb\_o} : \texttt{wb\_adr\_o}, \texttt{wb\_sel\_o} and \texttt{wb\_we\_o}.
        }

      \req{D\_LSM\_WISHBONE\_STALL\_01}{
          While initiating a request, the memory interface shall hold the state of its outputs until \texttt{wb\_stall\_i} is deasserted.
        }

      \vspace{0.5em}

      \begin{figure}[H]
          \centering
          {
  \vspace{0.5em}
  \begin{center}
    \refstepcounter{table}
    Table \thetable: Description of the single read cycle of the wishbone memory interface defined in figure \ref{fig:wishbone-single-read-cycle}.\label{tab:wishbone-single-read-cycle}
  \end{center}

\footnotesize
\begin{xltabular}{0.9\textwidth}{|l|X|}
  \hline
  \cellcolor{gray!20}\textbf{CLOCK EDGE} & \cellcolor{gray!20}\textbf{DESCRIPTION} \\
  \hline
  \multirow{5}{*}{0} & The memory interface presents a valid address on \texttt{adr\_o} \\
  & The memory interface deasserts \texttt{we\_o} to indicate a READ cycle \\
  & The memory interface presents a bank select \texttt{sel\_o} to indicate where it expects data. \\
  & The memory interface asserts \texttt{cyc\_o} to indicate the start of the cycle. \\
  & The memory interface asserts stb\_o to indicate the start of the phase. \\
  \hline
  \multirow{3}{*}{1} & Valid data is provided on \texttt{dat\_i}. \\
  & \texttt{ack\_i} is asserted to indicate valid data. \\
  & The memory interface deasserts \texttt{stb\_o} to indicate end of data phase. \\
  \hline
  \multirow{3}{*}{2} & The memory interface latches data on \texttt{dat\_i}. \\
  & The memory interface deasserts \texttt{cyc\_o} to indicate the end of the cycle. \\
  & \texttt{ack\_i} is deasserted. \\
  \hline
\end{xltabular}
}

          \caption{Timing diagram of the single read cycle of the wishbone memory interface}
          \label{fig:lsm-wishbone-single-read-cycle}
        \end{figure}

      {
  \vspace{0.5em}
  \begin{center}
    \refstepcounter{table}
    Table \thetable: Description of the single read cycle of the wishbone memory interface defined in figure \ref{fig:lsm-wishbone-single-read-cycle}.\label{tab:lsm-wishbone-single-read-cycle}
  \end{center}

\footnotesize
\begin{xltabular}{0.9\textwidth}{|l|X|}
  \hline
  \cellcolor{gray!20}\textbf{CLOCK EDGE} & \cellcolor{gray!20}\textbf{DESCRIPTION} \\
  \hline
  \multirow{5}{*}{0} & The memory interface presents a valid address on \texttt{adr\_o} \\
  & The memory interface deasserts \texttt{we\_o} to indicate a READ cycle \\
  & The memory interface presents a bank select \texttt{sel\_o} to indicate where it expects data. \\
  & The memory interface asserts \texttt{cyc\_o} to indicate the start of the cycle. \\
  & The memory interface asserts stb\_o to indicate the start of the phase. \\
  \hline
  \multirow{3}{*}{1} & Valid data is provided on \texttt{dat\_i}. \\
  & \texttt{ack\_i} is asserted to indicate valid data. \\
  & The memory interface deasserts \texttt{stb\_o} to indicate end of data phase. \\
  \hline
  \multirow{3}{*}{2} & The memory interface latches data on \texttt{dat\_i}. \\
  & The memory interface deasserts \texttt{cyc\_o} to indicate the end of the cycle. \\
  & \texttt{ack\_i} is deasserted. \\
  \hline
\end{xltabular}
}


      \req{D\_LSM\_WISHBONE\_READ\_CYCLE\_01}{
          A read transaction shall be started by asserting both \texttt{wb\_cyc\_o} and \texttt{wb\_stb\_o}, and deasserting \texttt{wb\_we\_o}.
        }

      \req{D\_LSM\_WISHBONE\_READ\_CYCLE\_02}{
          The memory interface shall conform to the READ cycle detailed in figure \ref{fig:wishbone-single-read-cycle}.
        }

      \begin{figure}[H]
          \centering
          {
  \vspace{0.5em}
  \begin{center}
    \refstepcounter{table}
    Table \thetable: Description of the single write cycle of the wishbone memory interface defined in figure \ref{fig:wishbone-single-write-cycle}.\label{tab:wishbone-single-write-cycle}
  \end{center}

\footnotesize
\begin{xltabular}{0.9\textwidth}{|l|X|}
  \hline
  \cellcolor{gray!20}\textbf{CLOCK EDGE} & \cellcolor{gray!20}\textbf{DESCRIPTION} \\
  \hline
  \multirow{6}{*}{0} & The memory interface presents a valid address on \texttt{adr\_o} \\
  & The memory interface presents valid data on \texttt{dat\_o} \\
  & The memory interface asserts \texttt{we\_o} to indicate a WRITE cycle \\
  & The memory interface presents a bank select \texttt{sel\_o} to indicate where it sends data. \\
  & The memory interface asserts \texttt{cyc\_o} to indicate the start of the cycle. \\
  & The memory interface asserts stb\_o to indicate the start of the phase. \\
  \hline
  \multirow{1}{*}{1} & \texttt{ack\_i} is asserted in response to \texttt{stb\_o} to indicate latched data. It shall be noted that wait states may be inserted before asserting \texttt{ack\_i}, thereby allowing it to throttle the cycle speed. Any number of wait states may be added. \\
  \hline
  \multirow{2}{*}{2} & The memory interface deasserts \texttt{stb\_o} and \texttt{cyc\_o} to indicate the end of the cycle. \\
  & The \texttt{ack\_i} signal is deasserted in response to the deassertion of \texttt{stb\_o}.\\
  \hline
\end{xltabular}
}

          \caption{Timing diagram of the single write cycle of the wishbone memory interface}
          \label{fig:lsm-wishbone-single-write-cycle}
        \end{figure}

      {
  \vspace{0.5em}
  \begin{center}
    \refstepcounter{table}
    Table \thetable: Description of the single write cycle of the wishbone memory interface defined in figure \ref{fig:lsm-wishbone-single-write-cycle}.\label{tab:lsm-wishbone-single-write-cycle}
  \end{center}

\footnotesize
\begin{xltabular}{0.9\textwidth}{|l|X|}
  \hline
  \cellcolor{gray!20}\textbf{CLOCK EDGE} & \cellcolor{gray!20}\textbf{DESCRIPTION} \\
  \hline
  \multirow{6}{*}{0} & The memory interface presents a valid address on \texttt{adr\_o} \\
  & The memory interface presents valid data on \texttt{dat\_o} \\
  & The memory interface asserts \texttt{we\_o} to indicate a WRITE cycle \\
  & The memory interface presents a bank select \texttt{sel\_o} to indicate where it sends data. \\
  & The memory interface asserts \texttt{cyc\_o} to indicate the start of the cycle. \\
  & The memory interface asserts stb\_o to indicate the start of the phase. \\
  \hline
  \multirow{1}{*}{1} & \texttt{ack\_i} is asserted in response to \texttt{stb\_o} to indicate latched data. It shall be noted that wait states may be inserted before asserting \texttt{ack\_i}, thereby allowing it to throttle the cycle speed. Any number of wait states may be added. \\
  \hline
  \multirow{2}{*}{2} & The memory interface deasserts \texttt{stb\_o} and \texttt{cyc\_o} to indicate the end of the cycle. \\
  & The \texttt{ack\_i} signal is deasserted in response to the deassertion of \texttt{stb\_o}.\\
  \hline
\end{xltabular}
}


      \req{D\_LSM\_WISHBONE\_WRITE\_CYCLE\_01}{
          A write transaction shall be started by asserting both \texttt{wb\_cyc\_o}, \texttt{wb\_stb\_o} and \texttt{wb\_we\_o}.
        }

      \req{D\_LSM\_WISHBONE\_WRITE\_CYCLE\_02}{
          The memory interface shall conform to the WRITE cycle detailed in figure \ref{fig:wishbone-single-write-cycle}.
        }

      \vspace{0.5em}

      \req{D\_LSM\_WISHBONE\_TIMING\_01}{
          The clock input \texttt{clk\_i} shall coordinate all activites for the internal logic within the memory interface. All output signals of the memory interface shall be registered at the rising edge of \texttt{clk\_i}. All input signals of the memory interface shall be stable before the rising edge of \texttt{clk\_i}.
        }[
          rationale={As long as the memory interface is designed within the clock domain of \texttt{clk\_i}, the requirement will be satisfied by using the place and route tool.}
        ]

      \begin{content}
          BLOCK cycles are not supported in revision 1.0.0.
        \end{content}

    \subsubsection{Output}

      \paragraph{Register write}

        \req{D\_LSM\_OUTPUT\_WRITE\_01}{
            The \texttt{reg\_write\_o} output shall be set to the value of \texttt{reg\_write\_i} on the rising edge of \texttt{clk\_i} when both \texttt{input\_ready\_o} and \texttt{input\_valid\_i} are asserted.
          }

        \req{D\_LSM\_OUTPUT\_WRITE\_02}{
            The \texttt{reg\_write\_o} output shall be deasserted on the rising edge of \texttt{clk\_i} when \texttt{input\_ready\_o} is asserted and \texttt{input\_valid\_i} is deasserted.
          }

      \paragraph{Register addr}

        \req{D\_LSM\_OUTPUT\_ADDR\_01}{
            The \texttt{reg\_addr\_o} output shall be set to the value of \texttt{reg\_addr\_i} on the rising edge of \texttt{clk\_i} when \texttt{input\_ready\_o} is asserted.
          }

      \paragraph{Register data}

        \req{D\_LSM\_OUTPUT\_DATA\_01}{
            The \texttt{reg\_data\_o} shall be set to the value of \texttt{wb\_dat\_i} on the rising edge of \texttt{clk\_i} when \texttt{wb\_ack\_i} is asserted.
          }

        \req{D\_LSM\_OUTPUT\_DATA\_02}{
            The \texttt{reg\_data\_o} shall be set to the value of \texttt{alu\_result\_i} on the rising edge of \texttt{clk\_i} when \texttt{input\_ready\_o} is asserted.
          }

    \subsubsection{Output handshake}

      \req{D\_LSM\_OUTPUT\_HANDSHAKE\_01}{
          The signal \texttt{output\_valid\_o} shall be asserted on the rising edge of \texttt{clk\_i} when leaving the \texttt{DONE} state.
        }

      \req{D\_LSM\_OUTPUT\_HANDSHAKE\_02}{
          The \texttt{output\_valid\_i} signal shall be asserted on the rising edge of \texttt{clk\_i} when \texttt{input\_ready\_i} is asserted and \texttt{enable\_i} is deasserted. 
        }

\newpage
