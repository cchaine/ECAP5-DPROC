\section{Instruction Fetch Module}

  \subsection{Interface}

    \begin{content}
        The instruction fetch module handles fetching from memory the instructions to be executing. The signals are described in table \ref{tab:ifm-interface}. 
      \end{content}

    {
  \vspace{0.5em}
  \begin{center}
    \refstepcounter{table}
    Table \thetable: Instruction Fetch Module interface signals\label{tab:ifm-interface}
  \end{center}

\footnotesize
\begin{xltabular}{0.9\textwidth}{|l|c|c|X|}
  \hline
  \cellcolor{gray!20}\textbf{NAME} & \cellcolor{gray!20}\textbf{TYPE} & \cellcolor{gray!20}\textbf{WIDTH} & \cellcolor{gray!20}\textbf{DESCRIPTION} \\
  \hline
  clk\_i & I & 1 & Clock input. \\
  \hline
  rst\_i & I & 1 & Reset input. \\
  \hline
  \multicolumn{4}{|l|}{\textbf{JUMP LOGIC}} \\
  \hline
  irq\_i & I & 1 & External interrupt request. \\
  \hline
  drq\_i & I & 1 & External debug request. \\
  \hline
  branch\_i & I & 1 & Branch request. \\
  \hline
  boffset\_i & I & 20 & Branch offset from the pc of the instruction in the execute stage (pc - 8).  \\
  \hline
  \multicolumn{4}{|l|}{\textbf{WISHBONE MASTER}} \\
  \hline
  wb\_adr\_o & O & 32 & The address output array is used to pass binary address. \\
  \hline
  wb\_dat\_i & I & 32 & The data input array is used to pass binary data. \\
  \hline
  wb\_we\_o & O & 1 & The write enable output indicates whether the current local bus cycle is a READ or WRITE cycle. This signal is negated during READ cycles and is asserted during WRITE cycles. This signal is always low as this interface only supports READ cycles. \\
  \hline
  wb\_sel\_o & O & 4 & The select output array indicates where valid data is expected on the wb dat i signal array during READ cycles. Each individual select signal correlates to one of four active bytes on the 32-bit data port. \\
  \hline
  wb\_stb\_o & O & 1 & The strobe output indicates a valid data transfer cycle. It is used to qualify various other signals on the interface. \\
  \hline
  wb\_ack\_i & I & 1 & The acknowledge input, when asserted, indicates the nor- mal termination of a bus cycle. \\
  \hline
  wb\_cyc\_o & O & 1 & The cycle output, when asserted, indicates that a valid bus cycle is in progress. This signal is asserted for the duration of all bus cycles. \\
  \hline
  wb\_stall\_i & I & 1 & The pipeline stall input indicates that current slave is not able to accept the transfer in the transaction queue. \\
  \hline
  \multicolumn{4}{|l|}{\textbf{OUTPUT LOGIC}} \\
  \hline
  output\_ready\_i & I & 1 & Output handshaking signal asserted when the destination is ready to receive the output \\
  \hline
  output\_valid\_o & O & 1 & Output handshaking signal asserted when the output is valid. \\
  \hline
  instr\_o & O & 32 & Instruction output. \\
  \hline
\end{xltabular}
}


  \subsection{Specification}

    \subsubsection{State machine}

      \begin{figure}[H]
          \centering
          \vspace{1em}
\begin{tikzpicture}[->, >=stealth, node distance=3cm, every state/.style={thick, fill=gray!20}, initial text=$ $]
    \node[state,minimum width=2cm,initial]  (idle)            {\texttt{IDLE}};
    \node[state,minimum width=2cm, align=center]          (memory-stall)  [below right of=idle]  {\footnotesize{\texttt{MEMORY}} \\ \footnotesize{\texttt{STALL}}};
    \node[state,minimum width=2cm, align=center]          (request)       [above right of=memory-stall]  {\footnotesize{\texttt{REQUEST}}};
    \node[state,minimum width=2cm, align=center]          (memory-wait)   [below right of=request]  {\footnotesize{\texttt{MEMORY}} \\ \footnotesize{\texttt{WAIT}}};
    \node[state,minimum width=2cm, align=center]          (done)          [above right of=memory-wait]  {\texttt{DONE}};
    \node[state,minimum width=2cm, align=center]          (pipeline-stall) [right of=done, xshift=1cm] {\footnotesize{\texttt{PIPELINE}} \\ \footnotesize{\texttt{STALL}}};

  \draw (idle)            edge              (memory-stall)  
        (idle)            edge              (request)                         
        (memory-stall)    edge              (request)                         
        (request)         edge              (memory-wait)                     
        (request)         edge              (done)                            
        (memory-wait)     edge              (done)                            
        (done)            edge[bend right]  (idle)          
        (done)            edge              (pipeline-stall)                  
        (pipeline-stall)  edge[bend right]  (idle);
\end{tikzpicture}

          \caption{State diagram of the instruction fetch module}
          \label{fig:ifm-state-diagram}
        \end{figure}

      \req{D\_IFM\_STATE\_01}{
          The IFM shall be in the \texttt{IDLE} (\texttt{0}) state while \texttt{rst\_i} is asserted or on a successfull output handshake.
        }

      \req{D\_IFM\_STATE\_02}{
          The IFM shall be in the \texttt{REQUEST} (\texttt{1}) state when a memory request is successfully sent.
        }

      \req{D\_IFM\_STATE\_03}{
          The IFM shall be in the \texttt{MEMORY\_WAIT} (\texttt{2}) state when a memory request has been sent but the response has not been received.
        }

      \req{D\_IFM\_STATE\_04}{
          The IFM shall be in the \texttt{DONE} (\texttt{3}) state when a memory request response is successfully received.
        }

      \req{D\_IFM\_STATE\_05}{
          The IFM shall be in the \texttt{MEMORY\_STALL} (\texttt{4}) state when a memory request is ready but the memory interface is stalled.
        }

      \req{D\_IFM\_STATE\_06}{
          The IFM shall be in the \texttt{PIPELINE\_STALL} (\texttt{5}) state when a memory request output is ready but the pipeline is stalled.
        }

    \subsubsection{PC register}

      \req{D\_IFM\_PC\_REGISTER\_01}{
          The instruction fetch module shall implement a \texttt{pc} register which shall store the address of the instruction to be fetched.
        }[
          derivedfrom=F\_REGISTERS\_03
        ]

      \req{D\_IFM\_PC\_LOAD\_01}{
          The value of the \texttt{pc} register shall be loaded with \texttt{IRQ\_HANDLER\_ADDR} on the rising edge of \texttt{clk\_i} when \texttt{irq\_i} is asserted.
        }[
          derivedfrom=F\_IRQ\_HANDLER\_01
        ]

      \req{D\_IFM\_PC\_LOAD\_02}{
          The value of the \texttt{pc} register shall be loaded with \texttt{DRQ\_HANDLER\_ADDR} on the rising edge of \texttt{clk\_i} when \texttt{drq\_i} is asserted.
        }[
          derivedfrom=F\_DRQ\_HANDLER\_01
        ]

      \req{D\_IFM\_PC\_LOAD\_03}{
          The value of the \texttt{pc} register shall be incremented by 4 on the rising edge of \texttt{clk\_i} after both \texttt{output\_ready\_i} and \texttt{output\_valid\_o} where asserted.
        }[
          derivedfrom=F\_REGISTERS\_03,
          rationale=Instructions are stored consecutively and aligned on a four-byte boundary.
        ]

      \req{D\_IFM\_PC\_LOAD\_04}{
        The value of the \texttt{pc} register shall be incremented by the value of \texttt{boffset\_i} on the rising edge of \texttt{clk\_i} when \texttt{branch\_i} is asserted. \texttt{pc} shall not be updated on subsequent asserted \texttt{branch\_i} until the pending memory requests are terminated.
        }[
          derivedfrom={F\_JAL\_02, F\_JALR\_02, F\_BEQ\_02, F\_BNE\_02, F\_BLT\_02, F\_BGE\_02, F\_BLTU\_02, F\_BGEU\_02}
        ]

      \req{D\_IFM\_PC\_REGISTER\_02}{
          Any changes to the \texttt{pc} register shall be performed before any memory requests.
        }

      \req{D\_IFM\_PC\_PRECEDENCE\_01}{
          In the case where multiple load requirement can apply, the following precedence shall be applied, starting with the highest: \reqref{D\_IFM\_PC\_LOAD\_02}, \reqref{D\_IFM\_PC\_LOAD\_01}, \reqref{D\_IFM\_PC\_LOAD\_04} and \reqref{D\_IFM\_PC\_LOAD\_03}.
        }

    \subsubsection{Reset}

      \req{D\_IFM\_RESET\_01}{
          The value of the \texttt{pc} register shall be loaded with \texttt{BOOT\_ADDR} on the rising edge of \texttt{clk\_i} following the assertion of \texttt{rst\_i}.
        }[
          derivedfrom=F\_REGISTERS\_RESET\_01
        ]

      \req{D\_IFM\_RESET\_02}{
          The \texttt{output\_valid\_o} signal shall be deasserted on the rising edge of \texttt{clk\_i} following the assertion of \texttt{rst\_i}.
        }

    \subsubsection{Fetch triggering}

      \event{EV\_IFM\_FETCH\_REQUEST\_01}{
          This event represents an instruction fetch request.
        }

      \req{D\_IFM\_FETCH\_TRIGGER\_01}{
          The instruction fetch module shall trigger \eventref{EV\_IFM\_FETCH\_REQUEST\_01} on the rising edge of \texttt{clk\_i} following the deassertion of \texttt{rst\_i}.
        }[
          derivedfrom=A\_INSTRUCTION\_FETCH\_01
        ]

      \req{D\_IFM\_FETCH\_TRIGGER\_02}{
          The instruction fetch module shall trigger \eventref{EV\_IFM\_FETCH\_REQUEST\_01} on the rising edge of \texttt{clk\_i} after both \texttt{output\_ready\_i} and \texttt{output\_valid\_o} are asserted.
        }[
          derivedfrom=A\_INSTRUCTION\_FETCH\_01
        ]

      \req{D\_IFM\_MEMORY\_FETCH\_01}{
          A 32-bit read cycle at the address represented by \texttt{pc} shall be initiated when \eventref{EV\_IFM\_FETCH\_REQUEST\_01} is triggered.
        }

    \subsubsection{Fetch cancellation}
      
      \event{EV\_IFM\_REQUEST\_CANCEL\_01}{
          This event represents the cancellation of any pending memory requests
        }

      \req{D\_IFM\_REQUEST\_CANCEL\_01}{
          When \eventref{EV\_IFM\_REQUEST\_CANCEL\_01} is triggered, the memory interface shall wait for the termination of the any pending memory request and disable the module outputs.
        }

      \begin{content}
        In the case where \eventref{EV\_IFM\_REQUEST\_CANCEL\_01} is triggered on the rising edge of \texttt{clk\_i} with a successfull output handshake, the instruction is still outputed. This might cause a problem which will be handled in HZDM. TBC
      \end{content}

      \req{D\_IFM\_REQUEST\_CANCEL\_02}{
        After \eventref{EV\_IFM\_REQUEST\_CANCEL\_01} has been triggered, \eventref{EV\_IFM\_FETCH\_REQUEST\_01} shall be triggered once all pending memory requests are terminated.
        }

      \req{D\_IFM\_REQUEST\_CANCEL\_03}{
          On the rising edge of \texttt{clk\_i} following the assertion of \texttt{drq\_i}, \eventref{EV\_IFM\_REQUEST\_CANCEL\_01} shall be triggered.
        }[
          derivedfrom=I\_DRQ\_01
        ]

      \req{D\_IFM\_REQUEST\_CANCEL\_04}{
          On the rising edge of \texttt{clk\_i} following the assertion of \texttt{irq\_i}, \eventref{EV\_IFM\_REQUEST\_CANCEL\_01} shall be triggered.
        }[
          derivedfrom=I\_IRQ\_01
        ]

      \req{D\_IFM\_REQUEST\_CANCEL\_05}{
          On the rising edge of \texttt{clk\_i} following the assertion of \texttt{branch\_i}, \eventref{EV\_IFM\_REQUEST\_CANCEL\_01} shall be triggered.
        }

    \subsubsection{Wishbone interface}

      \begin{content}
          The following requirements are extracted from the Wishbone specification for implementing the memory interface of the instruction fetch module.
        \end{content}

      \req{D\_IFM\_WISHBONE\_DATASHEET\_01}{
          The memory interface shall comply with the Wishbone Datasheet provided in section \ref{user-needs}.
        }

      \req{D\_IFM\_WISHBONE\_RESET\_01}{
          The memory interface shall initialize itself at the rising edge of \texttt{clk\_i} following the assertion of \texttt{rst\_i}.
        }

      \req{D\_IFM\_WISHBONE\_RESET\_02}{
          The memory interface shall stay in the initialization state until the rising edge of \texttt{clk\_i} following the deassertion of \texttt{rst\_i}.
        }

      \req{D\_IFM\_WISHBONE\_RESET\_03}{
          Signals \texttt{wb\_stb\_o} and \texttt{wb\_cyc\_o} shall be deasserted while the memory interface is in the initialization state. The state of all other memory interface signals are undefined in response to a reset cycle.
        }

      \req{D\_IFM\_WISHBONE\_TRANSFER\_CYCLE\_01}{
          The memory interface shall assert \texttt{wb\_cyc\_o} for the entire duration of the memory access.
        }[
          rationale=TBC what wb\_cyc\_o does.
        ]

      \req{D\_IFM\_WISHBONE\_TRANSFER\_CYCLE\_02}{
          Signal \texttt{wb\_cyc\_o} shall be asserted no later than the rising edge of \texttt{clk\_i} that qualifies the assertion of \texttt{wb\_stb\_o}.
        }

      \req{D\_IFM\_WISHBONE\_TRANSFER\_CYCLE\_03}{
            Signal \texttt{wb\_cyc\_o} shall be deasserted no earlier than the rising edge of \texttt{clk\_i} that qualifies the deassertion of \texttt{wb\_stb\_o}.
        }

      \req{D\_IFM\_WISHBONE\_HANDSHAKE\_01}{
          The memory interface shall accept \texttt{wb\_ack\_i} signals at any time after a transaction is initiated.
        }

      \req{D\_IFM\_WISHBONE\_HANDSHAKE\_02}{
          The memory interface must qualify the following signals with \texttt{wb\_stb\_o} : \texttt{wb\_adr\_o}, \texttt{wb\_sel\_o} and \texttt{wb\_we\_o}.
        }

      \req{D\_IFM\_WISHBONE\_STALL\_01}{
          While initiating a request, the memory interface shall hold the state of its outputs until \texttt{wb\_stall\_i} is deasserted.
        }

      \vspace{0.5em}

      \begin{figure}[H]
          \centering
          {
  \vspace{0.5em}
  \begin{center}
    \refstepcounter{table}
    Table \thetable: Description of the single read cycle of the wishbone memory interface defined in figure \ref{fig:wishbone-single-read-cycle}.\label{tab:wishbone-single-read-cycle}
  \end{center}

\footnotesize
\begin{xltabular}{0.9\textwidth}{|l|X|}
  \hline
  \cellcolor{gray!20}\textbf{CLOCK EDGE} & \cellcolor{gray!20}\textbf{DESCRIPTION} \\
  \hline
  \multirow{5}{*}{0} & The memory interface presents a valid address on \texttt{adr\_o} \\
  & The memory interface deasserts \texttt{we\_o} to indicate a READ cycle \\
  & The memory interface presents a bank select \texttt{sel\_o} to indicate where it expects data. \\
  & The memory interface asserts \texttt{cyc\_o} to indicate the start of the cycle. \\
  & The memory interface asserts stb\_o to indicate the start of the phase. \\
  \hline
  \multirow{3}{*}{1} & Valid data is provided on \texttt{dat\_i}. \\
  & \texttt{ack\_i} is asserted to indicate valid data. \\
  & The memory interface deasserts \texttt{stb\_o} to indicate end of data phase. \\
  \hline
  \multirow{3}{*}{2} & The memory interface latches data on \texttt{dat\_i}. \\
  & The memory interface deasserts \texttt{cyc\_o} to indicate the end of the cycle. \\
  & \texttt{ack\_i} is deasserted. \\
  \hline
\end{xltabular}
}

          \caption{Timing diagram of the single read cycle of the wishbone memory interface}
          \label{fig:ifm-wishbone-single-read-cycle}
        \end{figure}

      {
  \vspace{0.5em}
  \begin{center}
    \refstepcounter{table}
    Table \thetable: Description of the single read cycle of the wishbone memory interface defined in figure \ref{fig:ifm-wishbone-single-read-cycle}.\label{tab:ifm-wishbone-single-read-cycle}
  \end{center}

\footnotesize
\begin{xltabular}{0.9\textwidth}{|l|X|}
  \hline
  \cellcolor{gray!20}\textbf{CLOCK EDGE} & \cellcolor{gray!20}\textbf{DESCRIPTION} \\
  \hline
  \multirow{5}{*}{0} & The memory interface presents a valid address on \texttt{adr\_o} \\
  & The memory interface deasserts \texttt{we\_o} to indicate a READ cycle \\
  & The memory interface presents a bank select \texttt{sel\_o} to indicate where it expects data. \\
  & The memory interface asserts \texttt{cyc\_o} to indicate the start of the cycle. \\
  & The memory interface asserts stb\_o to indicate the start of the phase. \\
  \hline
  \multirow{3}{*}{1} & Valid data is provided on \texttt{dat\_i}. \\
  & \texttt{ack\_i} is asserted to indicate valid data. \\
  & The memory interface deasserts \texttt{stb\_o} to indicate end of data phase. \\
  \hline
  \multirow{3}{*}{2} & The memory interface latches data on \texttt{dat\_i}. \\
  & The memory interface deasserts \texttt{cyc\_o} to indicate the end of the cycle. \\
  & \texttt{ack\_i} is deasserted. \\
  \hline
\end{xltabular}
}


      \req{D\_IFM\_WISHBONE\_READ\_CYCLE\_01}{
          A read transaction shall be started by asserting both \texttt{wb\_cyc\_o} and \texttt{wb\_stb\_o}, and deasserting \texttt{wb\_we\_o}.
        }

      \req{D\_IFM\_WISHBONE\_READ\_CYCLE\_02}{
          The memory interface shall conform to the READ cycle detailed in figure \ref{fig:wishbone-single-read-cycle}.
        }


      \vspace{0.5em}

      \begin{content}
          The memory write cycles are not implemented in this module as it shall only read data from memory.
        \end{content}

      \req{D\_IFM\_WISHBONE\_TIMING\_01}{
          The clock input \texttt{clk\_i} shall coordinate all activites for the internal logic within the memory interface. All output signals of the memory interface shall be registered at the rising edge of \texttt{clk\_i}. All input signals of the memory interface shall be stable before the rising edge of \texttt{clk\_i}.
        }[
          rationale={As long as the memory interface is designed within the clock domain of \texttt{clk\_i}, the requirement will be satisfied by using the place and route tool.}
        ]


      \begin{content}
          BLOCK cycles are not supported in revision 1.0.0.
        \end{content}

    \subsubsection{Output}

      \req{D\_IFM\_OUTPUT\_01}{
          The signal \texttt{instr\_o} shall be set to the value represented by \texttt{wb\_dat\_i} on the rising edge of \texttt{clk\_i} following the assertion of \texttt{wb\_ack\_i}
        }

      \req{D\_IFM\_OUTPUT\_02}{
          The signal \texttt{pc\_o} shall be set to the value of the \texttt{pc} register on the rising edge of \texttt{clk\_i} following the assertion of \texttt{wb\_ack\_i}
        }

    \subsubsection{Output handshake}

      \req{D\_IFM\_OUTPUT\_HANDSHAKE\_01}{
          The signal \texttt{output\_valid\_o} shall be deasserted on the rising edge of \texttt{clk\_i} when \eventref{EV\_IFM\_FETCH\_REQUEST\_01} is triggered.
        }[
          rationale=Refer to section \ref{control-hazard}.
        ]

      \req{D\_IFM\_OUTPUT\_HANDSHAKE\_02}{
          The signal \texttt{output\_valid\_o} shall be asserted on the rising edge of \texttt{clk\_i} following the assertion of \texttt{wb\_ack\_i}.
        }[
          rationale=Refer to section \ref{control-hazard}.
        ]

      \req{D\_IFM\_OUTPUT\_HANDSHAKE\_03}{
          When \texttt{output\_valid\_o} is asserted, the instruction fetch module shall hold the value of the \texttt{instr\_o} signal until the rising edge of \texttt{clk\_i} following the assertion of \texttt{output\_ready\_i}.
        }[
          rationale=Refer to section \ref{pipeline-stall}.
        ]

      \req{D\_IFM\_OUTPUT\_HANDSHAKE\_04}{
          When \texttt{output\_valid\_o} is asserted, the instruction fetch module shall hold the value of the \texttt{pc\_o} signal until the rising edge of \texttt{clk\_i} following the assertion of \texttt{output\_ready\_i}.
        }[
          rationale=Refer to section \ref{pipeline-stall}.
        ]

\newpage
