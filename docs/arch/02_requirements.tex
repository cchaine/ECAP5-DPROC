\section{Overall Description}

\subsection{User needs}

\begin{content}
ECAP5 is the primary user for ECAP5-DPROC. ECAP5-DPROC could however be used as a standalone RISC-V processor. The following requirements define the user needs. 
\end{content}

\ureq{U\_INSTRUCTION\_SET\_01}{
  ECAP5-DPROC shall implement the RV32I instruction set.
}

\begin{content}
  In order to improve the usability of ECAP5-DPROC, it shall have a \textit{von Neumann} architecture as it only requires one memory interface.
\end{content}

\ureq{U\_MEMORY\_INTERFACE\_01}{
  ECAP5-DPROC shall access both instructions and data through a unique memory interface.
}

\ureq{U\_MEMORY\_INTERFACE\_02}{
  ECAP5-DPROC's unique memory interface shall be compliant with the AXI-Lite specification.
}

\ureq{U\_RESET\_01}{
  ECAP5-DPROC shall provide a signal which shall hold ECAP5-DPROC in a reset state while asserted.
}

\begin{content}
The polarity of the reset signal mentionned in \texttt{U\_RESET\_01} is not specified by the user.
\end{content}

\ureq{U\_BOOT\_ADDRESS\_01}{
  The address at which ECAP5-DPROC jumps after the reset signal is deasserted shall be hardware-configurable.
}

\begin{content}
The address mentionned in \texttt{U\_BOOT\_ADDRESS\_01} can be either configured through hardware signals or can be selected at compile-time.
\end{content}

\ureq{U\_HARDWARE\_INTERRUPT\_01}{
  ECAP5-DPROC shall provide an signal which shall interrupt ECAP5-DPROC's execution flow while asserted.
}

\ureq{U\_HARDWARE\_INTERRUPT\_02}{
  ECAP5-DPROC shall jump to a software-configurable address when it is interrupted.
}

\begin{content}
The memory address at which ECAP5-DPROC shall jump to when interrupted is not specified by the user.
\end{content}

\ureq{U\_DEBUG\_01}{
  ECAP5-DPROC shall be compliant with the RISC-V External Debug Support specification.
}

\begin{content}
There is no performance goal required by ECAP5 for ECAP5-DPROC as ECAP5 is an educational platform.
\end{content}

\subsection{Assumptions and Dependencies}

\begin{content}
Describe what the assumptions for the product are : Targeting the ecp5 family, based around opensource toolchains.
\end{content}

\section{Requirements}

\subsection{External Interface Requirements}

\begin{content}
Describes the interface of the product.
\end{content}

\subsection{Functional Requirements}

\req{REQ\_Test\_01}{
  Requirement title
}{
  This is the rationale of the requirement
}{
  This is the refers to of the requirement
}

\subsection{Nonfunctional Requirements}

\begin{content}
These can be : performance, safety, security, usability, scalability.
\end{content}

\newpage
